\documentclass[twocolumn,11pt,a4paper,german]{article}
\usepackage[margin=1cm]{geometry}
\usepackage[ngerman]{babel}
\setlength{\columnsep}{0.7cm}
\title{\vspace{-35pt}Topfenkolatschen mit Heidelbeeren}
\author{Traditionell}
\date{Für 2 Backbleche}
\pagenumbering{gobble}
\begin{document}
\maketitle
\section*{Zutaten}
\subsection*{Für den Teig}
\begin{itemize}
\item 50g Butter
\item 300ml Milch
\item 1 Würfel frische Hefe
\item 1 Vanilleschote
\item 500g Mehl
\item 80g Zucker
\item 1 Prise Salz
\item abgeriebene Schale von einer Bio-Zitrone
\item 1 Ei
\end{itemize}
\subsection*{Für die Füllung}
\begin{itemize}
\item 50g Butter
\item 250g Magerquark
\item 80g Zucker
\item 2 Eigelbe
\item 1 schwach gehäufter EL Vanillepuddingpulver
\item 1 Msp. gemahlene Muskatblüte
\item 250g Heidelbeeren
\end{itemize}
\subsection*{Außerdem}
\begin{itemize}
\item Mehl für die Arbeitsfläche
\item Backpapier für die Bleche
\item 1 Eigelb und 2TL Zucker zum Bestreichen
\end{itemize}
\section*{1}
Für den Teig die Butter schmelzen und abkühlen lassen. Die Milch leicht erwärmen
und die Hefe darin auflösen. Die Vanilleschote längs aufschlitzen und das Mark
herauskratzen. Mehl in eine Schüssel sieben, Zucker, Salz, Vanillemark,
Zitronenschale, Ei sowie die Butter und die Hefemilch dazugeben. Alles mit den
Knethaken des elektrischen Handrührgerätes so lange verkneten, bis sich der Teig
von der Schüssel löst.
\section*{2}
Den Hefeteig zugedeckt an einem warmen Ort gehen lassen, bis sich sein Volumen
verdoppelt hat. Den Teig auf der leicht bemehlten Arbeitsfläche 5 Minuten
kräftig kneten und schlagen, bis er nicht mehr klebt. Weitere 30 Minuten gehen
lassen, bis er sein Volumen verdoppelt hat.
\section*{3}
Für die Füllung Butter schmelzen und etwas abkühlen lassen. Den Quark mit
Butter, Zucker, Eigelben, Puddingpulver, Muskatblüte und Zitronenschale
verrühren. Die Heidelbeeren abbrausen, verlesen und trockentupfen.
\section*{4}
Den Teig auf der bemehlten Arbeitsfläche ausrollen. Sollte er zu weich sein,
lässt er sich einfach mit den Händen ausziehen. 20 Quadrate von 10cm Seitenlänge
ausschneiden. In die Mitte jedes Quadrates je 1 Esslöffel Quarkfüllung und
einige Beeren geben. Die diagonal gegenüberliegenden Teigspitzen etwas
langziehen und verdrehen, so dass kuvertartige Tschen entstehen. Die Füllung
dabei nicht ganz bedecken.
\section*{5}
2 Backbleche mit Backpapier auslegen. Die Taschen vorsichtig auf die Bleche
legen und zugedeckt 10 Minuten gehen lassen.
\section*{6}
Den Backofen auf 200°C (Gas Stufe 3-4, Umluft 180°C) vorheizen. Eigelb mit 1
Esslöffel Wasser und dem Zucker verquirlen und die Taschen damit bepinseln. Die
Kolatschen auf der mittleren Schiene in 20 bis 30 Minuten goldbraun backen. Auf
dem Gitter abkühlen lassen.
\vfill
\noindent\hrulefill
\vfill
\noindent
\flushright
\begin{tabular}{rl}
	Vorbereitungszeit: & ca. 50 Minuten \\
	Ruhezeit: & ca. 1 Stunde \\
	Backzeit: & 20-30 Minuten
\end{tabular}
\end{document}
