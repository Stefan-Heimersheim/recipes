\documentclass[german,a4paper]{article}
\usepackage{color}
\usepackage{url}
\usepackage[ngerman]{babel}
\begin{document}
\noindent \begin{center}
\textsl{\huge{}Triple Chocolate Brownies}{\huge\par}
\par\end{center}

\noindent \begin{center}
\textsc{saftige Brownies mit dreierlei Schokolade}
\par\end{center}

\noindent \begin{center}
{\small{}von }\textit{\textcolor{blue}{\small{}\url{www.chefkoch.de}}}{\small\par}
\par\end{center}

\noindent \textbf{Zutaten }für 16 Stücke\textbf{}\\

\noindent %
\begin{tabular}{ll}
\textbf{Für den Teig} & \tabularnewline
125 g & Schokolade, bitter\tabularnewline
125 g & Butter\tabularnewline
2 & Eier\tabularnewline
125 g & Zucker\tabularnewline
4 Tropfen & Butter-Vanille-Aroma\tabularnewline
125 g & Mehl\tabularnewline
50 g & Kakaopulver\tabularnewline
125 ml & Sahne\tabularnewline
125 g & Schokolade, Vollmilch\tabularnewline
125 g & Schokolade, weiße\tabularnewline
\textbf{Für die Glasur} & \tabularnewline
75 g & Schokolade, bitter\tabularnewline
75 g & Schokolade, Vollmilch\tabularnewline
75 ml & Sahne\tabularnewline
\end{tabular}\\
\\
\\
\begin{tabular}{ll}
\texttt{\footnotesize{}Arbeitszeit} & \texttt{\footnotesize{}ca. 30 min.}\tabularnewline
\texttt{\footnotesize{}Kalorien p. P.} & \texttt{\footnotesize{}ca. 350 kcal}\tabularnewline
\end{tabular}\\
\\
\\
\textbf{1.}\\
Eine ca. 20 x 20 bzw. 28 x 15 cm große Form einfetten und kalt stellen.
Die Schokolade im Wasserbad schmelzen, die Butter unterrühren. Die
Vollmilch- und die weiße Schokolade hacken. Den Ofen inzwischen auf
200°c Ober-/Unterhitze vorheizen. Die Eier schaumig schlagen, den
Zucker einrieseln lassen und weiterschlagen, bis die Masse cremig
wird. Das Aroma unterrühren. Mehl und Kakao mischen, mit der Sahne
unterrühren die geschmolzene Schokolade und die Schokostücke unterheben.
In die vorbereitete Form füllen und im vorgeheizten Ofen etwa 25 min
backen, sodass ein Stäbchen bei der Stäbchenprobe noch leicht feucht
herauskommt. In der Form auskühlen lassen. 

\noindent \textbf{2.}\\
Dann die beiden Schokoladensorten im Wasserbad schmelzen und die Sahne
nach und nach darunter rühren. Diese Ganache über die ausgekühlten
Brownies streichen und fest werden lassen. Sind allerdings auch ohne
Guss \textendash noch warm- ein echter Genuss!

\end{document}

