%% LyX 2.3.1 created this file.  For more info, see http://www.lyx.org/.
%% Do not edit unless you really know what you are doing.
\documentclass[english]{article}
\usepackage[T1]{fontenc}
\usepackage[latin9]{inputenc}
\usepackage{textcomp}
\usepackage{amssymb}

\makeatletter

%%%%%%%%%%%%%%%%%%%%%%%%%%%%%% LyX specific LaTeX commands.
%% Because html converters don't know tabularnewline
\providecommand{\tabularnewline}{\\}

\@ifundefined{date}{}{\date{}}
\makeatother

\usepackage{babel}
\begin{document}
\noindent \begin{center}
\textsl{\huge{}Crostata al limone}{\huge\par}
\par\end{center}

\noindent \begin{center}
\textsc{Zitronenkuchen}
\par\end{center}

\noindent \begin{center}
{\small{}Italien (Kampanien)}{\small\par}
\par\end{center}

\noindent \textbf{Zutaten }für 12 Personen\textbf{}\\

\noindent %
\begin{tabular}{ll}
\textbf{Für den Teig} & \tabularnewline
200 g & Mehl\tabularnewline
4 EL & Zucker\tabularnewline
1 Päckchen & Vanillezucker\tabularnewline
1 Prise & Salz\tabularnewline
100 g & kalte Butter\tabularnewline
1 & Eigelb\tabularnewline
\textbf{Für den Belag} & \tabularnewline
4 & große Bio-Zitronen\tabularnewline
2 & Eier (Größe M)\tabularnewline
250 g & Zucker\tabularnewline
500 g & Ricotta\tabularnewline
2 EL & Puderzucker\tabularnewline
\end{tabular}\\
\\
\\
\begin{tabular}{ll}
\texttt{\footnotesize{}Zubereitungszeit:} & \texttt{\footnotesize{}1 Std.}\tabularnewline
\texttt{\footnotesize{}Kühlzeit:} & \texttt{\footnotesize{}30 Min.}\tabularnewline
\texttt{\footnotesize{}Backzeit:} & \texttt{\footnotesize{}45 Min.}\tabularnewline
\texttt{\footnotesize{}Pro Portion} & \texttt{\footnotesize{}ca. 330 kcal}\tabularnewline
\end{tabular}\\
\\
\\
Romantische Wege hich über dem Meer, zauberhafte Ausblicke über ganze
Haine voller duftender Zirtonenbäume - dafür ist die Amalfi-Küste
berühmt geworden. Und die Früchte dort sind tatsächlich eine Besonderheit:
viel größer als andere Zitronen Süditaliens, saftig und wunderbar
aromatisch. Vor allem die aus Minori\textbf{ }sind weitbekannt und
werden nicht nur in Süßspeisen verarbeitet, sondern auch zum feinen
Limoncello (Likör), den man eisgekühlt in tiefgefrosteten Gläsern
serviert.\textbf{}\\
\textbf{1.}\\
Für den Teig das Mehl mit dem Zucker, dem Vanillezucker und dem Salz
mischen und auf die Arbeitsfläche häufen. In der Mitte eine Mulde
formen. Butter in kleine Würfel schneiden und mit dem Eigelb in die
Mulde geben.\\
\textbf{2.}\\
Alles mit den Händen zu einem glatten geschmeidigen Teig verkneten.
Falls er zu trocken ist, tropfenweise kaltes Wasser unterarbeiten.
Den Teig zu einer Kugel formen.\textbf{}\\
\textbf{3.}\\
Teigkugel zwischen zwei Lagen Klarsichtfolie zu einer runden Teigplatte
in Größe der Form ausrollen. Eine Tarte- oder Springform (28 - 30
cm $\varnothing$) mit dem Teig auskleiden, einen 2 cm hohen Rand
hochziehen. Den Teig 30 Min. ins Gefrierfach stellen.\\
\textbf{4.}\\
Für den Belag 2 Zitronen heiß waschen und die Schale fein abreiben,
1 Zitrone auspressen\footnote{Alternativ kann für einen etwas saureren Geschmack auch die zweite
Zirtone ausgepresst werden.}. Die Eier mit 150 g Zucker gut schaumig schlagen, die Ricotta nach
und nach unterrühren. Die Zitronenschale und den Zitronensaft untermischen.\\
\textbf{5.}\\
Den Backofen auf 180°C vorheizen. Die Ricottamasse auf den Teigboden
häufen und gleichmäßig darauf verteilen. Crostata im Ofen (Mitte,
Umluft 160°C) etwa 45 Min. backen, bis sie schön gebräunt ist.\\
\textbf{6.}\\
Inzwischen die restlichen Zitronen heiß waschen und in sehr dünne
Scheiben schneiden, Scheiben vierteln. Übrigen Zucker mit 10 EL Wasser
erhitzen und kräftig aufkochen. Die Zitronenscheiben darin bei mittlerer
Hitze 5 Min. kochen, abkühlen lassen.\\
\textbf{7.}\\
Den Kuchen abkühlen lassen. Vor dem Servieren die Zitronenscheiben
abtropfen lassen und auf den Kuchen legen. Mit dem Puderzucker bestäuben.
Mit dem Flambierbrenner karamellisieren oder kurz unter die heißen
Grillschlangen schieben.
\end{document}
